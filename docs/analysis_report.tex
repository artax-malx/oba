\documentclass[english, 11pt, a4paper]{article}
\usepackage{amsmath, amssymb,amsthm}
\usepackage{setspace, natbib}
\usepackage{bm}
\usepackage{threeparttable}
\usepackage{graphicx}
\usepackage{booktabs}
\usepackage{dcolumn}
\usepackage{tabu}
\usepackage{longtable}
\usepackage{float}
\usepackage{caption}
\usepackage{subcaption}
\usepackage[procnames]{listings}
\usepackage{color}
\setlength{\textwidth}{15.5cm} \setlength{\textheight}{22cm}\setlength{\oddsidemargin}{-0.5mm}
\setlength{\parskip}{1ex plus0.5ex minus0.5ex}\setlength{\parindent}{0mm}
\DeclareMathOperator\erfc{erfc}
\usepackage{babel}


\definecolor{keywords}{RGB}{255,0,90}
\definecolor{comments}{RGB}{0,0,113}
\definecolor{red}{RGB}{160,0,0}
\definecolor{green}{RGB}{0,150,0}
 
\lstset{language=Python, 
        basicstyle=\ttfamily\small, 
        keywordstyle=\color{keywords},
        commentstyle=\color{comments},
        stringstyle=\color{red},
        showstringspaces=false,
        identifierstyle=\color{green},
        procnamekeys={def,class}}


\renewcommand{\bibfont}{\footnotesize}
\setlength{\bibsep}{1pt}

\begin{document}

\baselineskip18pt


\title{Order Book Analysis}

\author{Artagan Malsagov}

\date{\today}


\maketitle
%\newpage 
%\tableofcontents

%\section{Introduction}

\section{Data Description}

The data consist of 5 levels of both sides of the order book, for 5 different days.
Each days spans roughly 10 hours worth of data (36 billion micros, see table below)

\begin{table}[H]
    \centering
    \begin{tabular}{lrrrr}
    \toprule
    date & min timestamp & max timestamp & avg bbo mid & avg 5 level order volume \\
    \midrule
    20190610 & 0 & 36000000000 & 10064 & 673 \\
    20190611 & 0 & 36000000000 & 10127 & 866 \\
    20190612 & 0 & 36000000000 & 9999 & 955 \\
    20190613 & 0 & 36000000000 & 10065 & 908 \\
    20190614 & 0 & 35999621354 & 9894 & 797 \\
    \bottomrule
    \end{tabular}
    \label{tab2}
\end{table}


\subsection{Data resampling}
The order book is of the form below:

\begin{table}[H]
    \centering
    \begin{tabular}{crrrr}
    \toprule
    timestamp & bp0 & bq0 & ap0 & aq0 \\
    \midrule
    110 & 10045& 62 & 10055 & 98 \\
    175 & 10065& 46 & 10075 & 42 \\
    220 & 10075& 9 & 10080 & 25 \\
    \bottomrule
    \end{tabular}
    \label{tab2}
\end{table}

where the timestamps are in microseconds. Plotting a histogram of the frequency of the timestamps,
we see that the updates aren't uniformly distributed: 

 \begin{figure}[H] 
	\centering
	\includegraphics[width=0.90\textwidth]{../data/figures/hist_timestamps.png}
	\caption{Histogram of order update arrivals for date 20190610}
	\label{fig1}
\end{figure}

For the analysis the data will be resampled. Specifically, a grid will be used with a time interval of
100ms. This is done so as to reduce noise of the raw updates at microsecond level and detect any
signals in the data. Obviously, this grid size might cause information less and a more thorough
analysis can be done to optimize, but for practical considerations 100ms will be used as the
discretization step. 

Thus considering table \ref{tab2}, the timestamps will be rounded up to the nearest 10,000 micros.
The reason to round up is to avoid look-ahead bias when using the data as a trade signal, since
rounding down will match an order update with a timepoint in the past.

\section{Feature and Target Selection}
For the features and targets the following things are considered:
\begin{itemize}
    \item The simple average mid price:
        \begin{equation}
            P_{mid} = \frac{bp0 + ap0}{2}
        \end{equation}
    \item The inverse volume weighted mid price:
        \begin{equation}
            P_{mid} = \frac{bp0\times aq0 + ap0 \times bq0}{bq0 + aq0}
        \end{equation}
        This mid has the benefit of taking into account the order imbalance at the top level: if
        the buy order volume is higher, the price will be skewed higher to the ask, and vice-versa
        if the sell order volume is higher.
    \item The bid-offer spread


\end{itemize}

First a simple plot of the data. The simple mid price is considered:
 \begin{figure}[H] 
	\centering
	\includegraphics[width=0.90\textwidth]{../data/figures/time_series_mid_price_inv_mid_price.png}
	\caption{}
	\label{fig2}
\end{figure}

\section{Model Selection}

\section{Results}

%%\section{Conclusion}       

\bibliographystyle{plain}
%\bibliography{report_sci_comp_3}

\end{document}
